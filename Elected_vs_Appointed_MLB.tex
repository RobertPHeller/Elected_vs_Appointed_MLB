%%%%%%%%%%%%%%%%%%%%%%%%%%%%%%%%%%%%%%%%%%%%%%%%%%%%%%%%%%%%%%%%%%%%%%%%%%%%%
%
%  System        : 
%  Module        : 
%  Object Name   : $RCSfile$
%  Revision      : $Revision$
%  Date          : $Date$
%  Author        : $Author$
%  Created By    : Robert Heller
%  Created       : Sun Oct 28 11:08:11 2018
%  Last Modified : <181030.1312>
%
%  Description 
%
%  Notes
%
%  History
% 
%%%%%%%%%%%%%%%%%%%%%%%%%%%%%%%%%%%%%%%%%%%%%%%%%%%%%%%%%%%%%%%%%%%%%%%%%%%%%
%
%    Copyright (C) 2018  Robert Heller D/B/A Deepwoods Software
%			51 Locke Hill Road
%			Wendell, MA 01379-9728
%
%    This program is free software; you can redistribute it and/or modify
%    it under the terms of the GNU General Public License as published by
%    the Free Software Foundation; either version 2 of the License, or
%    (at your option) any later version.
%
%    This program is distributed in the hope that it will be useful,
%    but WITHOUT ANY WARRANTY; without even the implied warranty of
%    MERCHANTABILITY or FITNESS FOR A PARTICULAR PURPOSE.  See the
%    GNU General Public License for more details.
%
%    You should have received a copy of the GNU General Public License
%    along with this program; if not, write to the Free Software
%    Foundation, Inc., 675 Mass Ave, Cambridge, MA 02139, USA.
%
% 
%
%%%%%%%%%%%%%%%%%%%%%%%%%%%%%%%%%%%%%%%%%%%%%%%%%%%%%%%%%%%%%%%%%%%%%%%%%%%%%

\documentclass[12pt]{article}
\usepackage{times}
\usepackage{url}
\usepackage{graphicx}
\usepackage{mathptm}
\usepackage[pdftex,pagebackref=true]{hyperref}
\hypersetup{%
  colorlinks=true,%
  linkcolor=blue,%
  citecolor=blue,%
  unicode%
}
\title{Elected boards vs. Appointed boards: a point-by-point examination of why one or the other.}
\author{Robert Heller}
\date{\today}
\begin{document}

\maketitle

\tableofcontents

\section{Introduction}

There are several ``boards''\footnote{Some are called boards, some are
committees, and some are commissions. For the purpose of this paper, I am
considering these terms as interchangeable.} in town government. Some have
elected members and some have appointed members\cite{Classroom2017}. There are
various reasons for having some boards with elected members and some with
appointed members. Generally there are statutory requirements (state law or
town meeting votes) for some and not for others. Aside from these statutory
requirements, there are also underlying reasons. This paper will explore those
reasons.

\section{An appointed board's advantages}

There are some advantages to an appointed board, including avoiding the costs 
associated with an election, an appointed ad-hoc committee can be created with 
a simple Selectboard vote, and the Selectboard can simply select citizens (or 
even non-citizens) to fill out the membership of the board and often can 
select people with the necessary knowledge or expertise.

\section{An appointed board's disadvantages}

There are some disadvantages to an appointed board, including the Selectboard
not always knowing who might be available, willing, and knowledgeable to serve
on a given board. The Selectboard may not have the time or expertise to fully
vette candidates to serve on a given board. The board members are only
accountable to the Selectboard (and not to the public at large). The
electorate is \textbf{not} fully informed of the positions and platforms of
the board members. The board membership selection process can be influenced by
Selectboard politics.  It is also possible for the Selectboard to appoint a 
fresh board in a given year, which can be problematical for long term boards, 
because there is a lack of continuity.

\section{An Elected board's advantages}

There are some advantages to an elected board, including the electoral
process (from nomination to the election itself) opening up the selection of
candidates to the whole town. The candidate(s) are put before the whole
electorate to be scrutinized by a large number of people in a wide public
forum. The candidate(s) are often required or compelled to present position
paper(s) to fully explain their position and/or qualifications for the board
positions that they are running for. An Elected board is independent of
Selectboard politics. Sometimes this is necessary for a board to function
properly. The electorate is fully informed of the positions and platforms of
the board members and the electorate can hold the board members accountable. 
An elected board's members serve for multiple years and their terms are 
staggered, which allows for good continuity over time, which can be important 
for boards with long (indefinite) lifetimes.

\section{An Elected board's disadvantages}

There are some disadvantages to an  elected board, including the cost of an 
election. It is also possible that an unqualified candidate could run and be 
elected. 

\section{Conclusions}

Generally, an appointed board bypasses the democratic process. Sometimes this
is appropriate, particularly in the case of short term and/or specialized
advisory committees (where the democratic process just adds an unnecessary and
cumbersome process), but sometimes this is actually not appropriate,
particularly for long term policy making or regulatory boards, which are meant
to serve the public at large over the long term (for as long as the town
exists), rather than advise the Selectboard (or other elected body, such as
town meeting) over the short term. An elected board fully engages the town
through the democratic process. Elected board candidates both inform the town
of their positions and platforms and are in turn informed by the electorate of
what the electorate wants of the board. There is a fully open and complete
bi-direction communication between the voting public and the board membership.
This allows the voting public to have confidence in the board and allows the
board to do its job properly and satisfactorily.

It is noteworthy that MGL Chapter 164\cite{MGL164}, the law governing
Municipal Light Plants, says that cities and towns may create an elected
Municipal Light Board to oversee the Municipal Light Plant, and says nothing
about an appointed Municipal Light Board. Creating an appointed Municipal
Light Board is treading into mostly uncharted legal territory.

\clearpage
\addcontentsline{toc}{section}{References}
\bibliography{Elected_vs_Appointed_MLB}
\bibliographystyle{plain}
\end{document}

